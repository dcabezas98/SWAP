\documentclass{article}

\usepackage[left=2cm,right=2cm,top=2cm,bottom=2cm]{geometry} 

\usepackage[utf8]{inputenc}   % otra alternativa para los caracteres acentuados y la "ñ"
\usepackage[           spanish % para poder usar el español
                      ,es-tabla % para los captions de las tablas
                       ]{babel}   
\decimalpoint %para usar el punto decimal en vez de coma para los números con decimales

%\usepackage{beton}
%\usepackage[T1]{fontenc}

\usepackage{parskip}
\usepackage{xcolor}

\usepackage{caption}

\usepackage{fancyvrb}

\usepackage{enumerate} % paquete para poder personalizar fácilmente la apariencia de las listas enumerativas

\usepackage{graphicx} % figuras
\usepackage{subfigure} % subfiguras

\usepackage{amsfonts}
\usepackage{amsmath}

\usepackage[formats]{listings}
\lstdefineformat{R}{~=\( \sim \)}
\lstset{basicstyle=\ttfamily,format=R}

\definecolor{gris}{RGB}{220,220,220}
	
\usepackage{float} % para controlar la situación de los entornos flotantes

\restylefloat{figure}
\restylefloat{table} 
\setlength{\parindent}{0mm}


\usepackage[bookmarks=true,
            bookmarksnumbered=false, % true means bookmarks in 
                                     % left window are numbered
            bookmarksopen=false,     % true means only level 1
                                     % are displayed.
            colorlinks=true,
            allcolors=blue,
            urlcolor=blue]{hyperref}
\definecolor{webblue}{rgb}{0, 0, 0.5}  % less intense blue


\title{\Huge SWAP: Asegurar la granja web\vspace{10mm}}

\author{\huge David Cabezas Berrido \vspace{10mm} \\ 
  \huge dxabezas@correo.ugr.es \vspace{10mm}}

\begin{document}
\maketitle
\tableofcontents
\newpage

\section{Instalar un certificado SSL autofirmado para configurar el acceso por HTTPS}

Toda esta sección la haremos en la máquina 1.

Para habilitar el módulo SSL de Apache2, ejecutamos la siguiente línea.

\begin{Verbatim}[tabsize=4]
sudo a2enmod ssl 
\end{Verbatim}
Habilita el módulo y sus dependencias. Salida:
\begin{Verbatim}[tabsize=4]
Considering dependency setenvif for ssl:
Module setenvif already enabled
Considering dependency mime for ssl:
Module mime already enabled
Considering dependency socache_shmcb for ssl:
Enabling module socache_shmcb.
Enabling module ssl.
See /usr/share/doc/apache2/README.Debian.gz on how to configure SSL and create self-signed certificates.
To activate the new configuration, you need to run:
systemctl restart apache2
\end{Verbatim}
Restauramos el servicio con \verb^sudo systemctl restart apache2^. Ahora creamos una carpeta para los certificados de Apache, y creamos un par de clave y cerficiado. Le ponemos longitud de clave 2048 bits y 365 de validez.

\begin{figure}[H]
	\centering
	\includegraphics[width=180mm]{imgs/cert-create}
	\caption{Rellenamos los datos del certificado como se indica en el guión. Comprobamos que se ha creado el par correctamente.}
	\label{fig:cert-create}
\end{figure}

Como opciones avanzadas comentamos que \texttt{-x509} auto-firma el certificado, se obtendría una solicitud de certificado si no usásemos
esta opción. Además, la opción \\ \texttt{-subj "/C=TheCountry/CN=theCommonName/ST=theState/O=theOrganization/..."} permite especificar los datos
desde la orden, pueden consultarse las abreviaturas en \href{https://stackoverflow.com/questions/6464129/certificate-subject-x-509}
este post se encuentran los distintos atributos y sus abreviaturas.

Ahora modificamos el fichero de configuración \texttt{/etc/apache2/sites-available/default-ssl.conf}, tenemos que tener el siguiente
bloque (\verb|SSLEngine on| ya estaba puesto).

\begin{Verbatim}[tabsize=4]
#   SSL Engine Switch:
#   Enable/Disable SSL for this virtual host.
SSLEngine on
SSLCertificateFile /etc/apache2/ssl/apache_dxabezas.crt
SSLCertificateKeyFile /etc/apache2/ssl/apache_dxabezas.key
\end{Verbatim}
También tenemos que comentar las líneas que sobreescriben estas directivas más abajo. Guardamos los cambios y ejecutamos
\begin{Verbatim}[tabsize=4]
sudo a2ensite default-ssl
sudo service apache2 reload
\end{Verbatim}

Cuando accedemos a la página, nos avisa de que es insegura porque el certificado es auto-firmado. Debemos permitir la excepción en
el navegador o añadir \texttt{-k} con \texttt{curl}. Si le damos al candado junto a la dirección y a \texttt{More Information},
podemos visualizar el certificado que hemos creado.

\begin{figure}[H]
	\centering
	\includegraphics[width=140mm]{imgs/cert-view}
	\caption{Certificado con los datos que hemos creado.}
	\label{fig:cert-view}
\end{figure}

Como opciones avanzadas, mostramos como obtener el certificado sin ayuda del navegador, con \texttt{openssl}:
\begin{Verbatim}[tabsize=4]
openssl s_client -connect 192.168.56.101:443 -showcerts
\end{Verbatim}

También hay varias opciones adicionales en la configuración de Apache2 SSL. Se activan con
\begin{Verbatim}[tabsize=4]
SSLOptions +opcion1 +opcion2
\end{Verbatim}

Por ejemplo, cuando se trabaja con autenticación y se requiere que los clientes también tengan certificados, la opción
\texttt{FakeBasicAuth} requiere que los clientes pongan el campo Subject the su certificado como usuario, la contraseña siempre
es la misma: ``xxj31ZMTZzkVA'' (que es una encriptación por DES de la palabra ``password''), por ello el nombre de Fake.

\section{Configurar la granja web para que permita HTTPS}

Primero replicamos la configuración en la máquina 2. En lugar de crear un nuevo certificado, copiamos el de M1.
Desde M1:
\begin{Verbatim}
sudo scp /etc/apache2/ssl/apache_dxabezas.crt dxabezas@192.168.56.102:/home/dxabezas/apache_dxabezas.crt
sudo scp /etc/apache2/ssl/apache_dxabezas.key dxabezas@192.168.56.102:/home/dxabezas/apache_dxabezas.key
\end{Verbatim}

Ahora desde M2:
\begin{Verbatim}
sudo mkdir /etc/apache2/ssl
sudo mv /home/dxabezas/apache_dxabezas.* /etc/apache2/ssl/
sudo a2enmod ssl & sudo service apache2 restart
sudo nano /etc/apache2/sites-available/default-ssl.conf
\end{Verbatim}

Escribimos las líneas 
\begin{Verbatim}
SSLCertificateFile /etc/apache2/ssl/apache_dxabezas.crt
SSLCertificateKeyFile /etc/apache2/ssl/apache_dxabezas.key
\end{Verbatim}
y comentamos las equivalentes que ya había. Terminamos con 
\begin{Verbatim}[tabsize=4]
sudo a2ensite default-ssl
sudo service apache2 reload
\end{Verbatim}

Cuando nos conectamos a M2 desde el navegador por HTTPS obtenemos el mismo resultado, y también podemos visualizar el mismo
certificado.

Falta configurar M3. Otra vez desde M1 copiamos el certificado:

\begin{Verbatim}
sudo scp /etc/apache2/ssl/apache_dxabezas.crt dxabezas@192.168.56.103:/home/dxabezas/apache_dxabezas.crt
sudo scp /etc/apache2/ssl/apache_dxabezas.key dxabezas@192.168.56.103:/home/dxabezas/apache_dxabezas.key
\end{Verbatim}

Y desde M3:
\begin{Verbatim}
mv apache_dxabezas.* ssl/
sudo nano /etc/nginx/conf.d/default.conf
\end{Verbatim}

Y creamos un nuevo server como se indica en el guión:

\begin{Verbatim}[tabsize=4]
server {
	listen 443 ssl;
	ssl on;
	ssl_certificate         /home/dxabezas/ssl/apache_dxabezas.crt;
	ssl_certificate_key     /home/dxabezas/ssl/apache_dxabezas.key;
	server_name balanceador_dxabezas;
	access_log /var/log/nginx/balanceador_dxabezas.access.log;
	error_log /var/log/nginx/balanceador_dxabezas.error.log;
	root /var/www/;
	
	location /
	{
		proxy_pass http://balanceo_dxabezas;
		proxy_set_header Host $host;
		proxy_set_header X-Real-IP $remote_addr;
		proxy_set_header X-Forwarded-For $proxy_add_x_forwarded_for;
		proxy_http_version 1.1;
		proxy_set_header Connection "";
	}
}
\end{Verbatim}
Tras restaurar NGINX, podemos acceder desde el navegador por HTTPS a la IP de M3 de la misma forma que hacíamos con
M1 y M2 (se marca como insegura y hay que añadir una excepción). Una vez entramos, M1 y M2 se turnan para servirnos
su \texttt{index.html}. También podemos ver el mismo certificado tal y como hacíamos antes.

Al igual que antes, todas las máquinas aceptan también tráfico HTTP.

Como opciones avanzadas, comentaremos \href{http://nginx.org/en/docs/http/ngx_http_ssl_module.html#ssl_protocols}{protocolos} y 
\href{http://nginx.org/en/docs/http/ngx_http_ssl_module.html#ssl_ciphers}{cifrados} para SSL. 

Dentro del archivo de configuración del server (NGINX), la directiva \texttt{ssl\_protocols} limita las conexiones a las compatibles con
las versiones de SSL y TLS que indiquemos de la siguiente lista: SSLv2, SSLv3, TLSv1, TLSv1.1, TLSv1.2, TLSv1.3.
Por defecto, las versiones que se aceptan son las de este ejemplo:
\begin{Verbatim}
ssl_protocols TLSv1 TLSv1.1 TLSv1.2;
\end{Verbatim}

Igualmente, la directiva \texttt{ssl\_ciphers} limita las conexiones a las compatibles con los sistemas de cifrado que se especifiquen,
deben escribirse en un formato entendiblel por OpenSSL, por ejemplo:
\begin{Verbatim}
ssl_ciphers ALL:!aNULL:!EXPORT56:RC4+RSA:+HIGH:+MEDIUM:+LOW:+SSLv2:+EXP;
\end{Verbatim}

\section{Certificados para Apache y NGINX con Certbot}

Instalamos Certbot en M1 y M3 con
\begin{Verbatim}
sudo apt install certbot
\end{Verbatim}

\subsection{Certificado para Apache}

Es necesario instalar el plugin de Apache en M1. Ejecutamos:
\begin{Verbatim}
sudo apt install python-certbot-apache
\end{Verbatim}

Ahora, siguiendo las instrucciones de \href{https://certbot.eff.org/lets-encrypt/ubuntubionic-apache}{esta página},
podemos obtener e instalar un certificado utilizando la siguiente orden.
\begin{Verbatim}
sudo certbot --apache
\end{Verbatim}

Nos pide algunos datos que debemos rellenar.

\begin{figure}[H]
	\centering
	\includegraphics[width=160mm]{imgs/certbot-apache}
	\caption{Rellenamos los datos para generar e instalar un certificado para que un sitio web acepte tráfico HTTPS.
	Este comando genera el certificado y realiza la configuración necesaria.}
	\label{fig:certbot-apache}
\end{figure}

Sin embargo, Certbot sólo genera certificados para nombres de dominio, lo que dificulta en gran medida que llevemos esto a la práctica.

Como opciones avanzadas, en la misma página explica qué debemos hacer si sólo queremos generar el certificado y queremos llevar a
cabo la configuración a mano.
\begin{Verbatim}
sudo certbot certonly --apache 
\end{Verbatim}
También es interesante comentar que por defecto Certbot instala un comando Crontab que renueva los certificados automáticamente.
En nuestro caso, nos ha creado el siguiente archivo.

\begin{figure}[H]
	\centering
	\includegraphics[width=140mm]{imgs/renew-crontab}
	\label{fig:renew-crontab}
\end{figure}

\subsection{Certificado para NGINX}

Primero instalamos (en M3) el plugin de NGINX.

\begin{Verbatim}
sudo apt install python-certbot-nginx
\end{Verbatim}

Ahora, siguiendo las instrucciones de \href{https://certbot.eff.org/lets-encrypt/ubuntubionic-nginx}{esta página}, ejecutamos:
\begin{Verbatim}
sudo certbot --nginx
\end{Verbatim}
Y nos pide los mismos datos que para el de Apache.

También podemos usar 
\begin{Verbatim}
sudo certbot certonly --nginx 
\end{Verbatim}
si sólo queremos generar el certificado para configurarlo manualmente.

\section{Cortafuegos \emph{iptables}}

\textbf{Nota:} Las IP de las máquinas 1 y 2 son 192.168.56.102 y 192.168.56.101 en lo que sigue.

El cortafuegos \emph{iptables} ya viene instalado en todas las máquinas.

Para configurarlo debemos escribir primero las reglas más generales y después las más específicas, ya que
tiene prioridad la última regla que se ejecuta (como es lógico). Comenzamos con un script que es equivalente
a desactivar el cortafuegos, borra todas las reglas y permite todo el tráfico.

\begin{Verbatim}
#!/bin/sh

# off.sh
# Eliminar todas las reglas, permitir todo el tráfico

iptables -F
iptables -X
iptables -Z
iptables -t nat -F
iptables -t nat -X
iptables -t nat -Z
iptables -t mangle -F
iptables -t mangle -X
iptables -t mangle -Z

iptables -P INPUT ACCEPT
iptables -P OUTPUT ACCEPT
iptables -P FORWARD ACCEPT
\end{Verbatim}

Una cadena es una lista de reglas que se aplican a una serie de paquetes.

La opción \texttt{-F} elimina todas las reglas de todas las cadenas (ya que no se ha especificado
ninguna cadena en contreto), \texttt{-X} elimina las cadenas y \texttt{-Z} pone a cero los contadores de bytes
transmitidos.

La opción \texttt{-t} indica la tabla a usar. Las distintas tablas (presentes en la configuración por defecto)
pueden encontrarse en \href{https://linux.die.net/man/8/iptables}{el manual}. Si no se indica nada, se trabaja
con la tabla \texttt{filter}, que cuenta (si no se añaden más) con tres cadenas (no se borran con -X): INPUT,
FORWARD y OUPUT. Cada tipo de paquete cae en una tabla y una cadena.

La opción \texttt{-P} sirve para establecer una política para una cadena.

Ahora presentamos un script que deniega todo el tráfico.

\begin{Verbatim}
#!/bin/sh

# denegacion.sh
# Denegación de todo el tráfico

iptables -P INPUT DROP
iptables -P OUTPUT DROP
iptables -P FORWARD DROP
\end{Verbatim}

El siguiente script contiene la configuración básica de un sitio web. Denegando todo el tráfico
excepto en los puertos correspondientes de los servicios que queremos mantener activos: SSH, HTTP y HTTPS.

\begin{Verbatim}
#!/bin/sh

# basic.sh
# Configuración básica de un sitio web
# Denegar todo excepto SSH, HTTP y HTTPS

# 1: Eliminar todas las reglas y permitir todo el tráfico, configuración limpia
./off.sh

# 2: Denegación implícita
./denegacion.sh

# 3: Permitir conexiones, puesto que somos un servidor web
iptables -A INPUT -m state --state NEW,ESTABLISHED,RELATED -j ACCEPT

# 4: Permitir acceso desde localhost (interface lo):
iptables -A INPUT -i lo -j ACCEPT
iptables -A OUTPUT -o lo -j ACCEPT

# 5: Abrir puerto 22 para permitir el acceso por SSH
iptables -A INPUT -p tcp --dport 22 -j ACCEPT
iptables -A OUTPUT -p tcp --sport 22 -j ACCEPT

# 6: Permitir tráfico por el puerto 80 (HTTP)
iptables -A INPUT -p tcp --dport 80 -j ACCEPT
iptables -A OUTPUT -p tcp --sport 80 -j ACCEPT

# 7: Permitir tráfico por el puerto 443 (HTTPS)
iptables -A INPUT -p tcp --dport 443 -j ACCEPT
iptables -A OUTPUT -p tcp --sport 443 -j ACCEPT
\end{Verbatim}

La opción \texttt{-A} añade nuevas reglas a una cadena. Por ejemplo, los paquetes entrantes (INPUT), transmitidos
por protocolo (opción \texttt{-p}) TCP con puerto de destino (\texttt{--dport}) 22 caerían en la regla debajo de
\#5.
La opción \texttt{-m state} indica que el estado debe ser uno de los especificados,
y la opción \texttt{-j} indica la acción a aplicar a los paquetes que correspondan con la regla (ACCEPT, DROP, REJECT,\ldots)
Las opciones \texttt{-i} y \texttt{-o} indican las interfaces de entrada y salida respectivamente.

\begin{figure}[H]
	\centering
	\includegraphics[width=140mm]{imgs/basic}
	\caption{Con esta configuración, la máquina 1 acepta tráfico SSH (no mostrado), HTTP y HTTPS, pero observamos
	que responde a un PING.}
	\label{fig:basic}
\end{figure}

Sólo tenemos que copiar este script en las distintas máquinas, y ya tenemos nuestra configuración básica.
\begin{Verbatim}
scp * dxabezas@192.168.56.101:/home/dxabezas/.iptables/
\end{Verbatim}
Ya lo tenemos también en M2, y 192.168.56.103 para M3. Hay que crear el directorio primero y desactivar el
cortafuegos, ya que la configuración que hemos hecho no permite que la máquina establezca conexiones por SSH, sólo que las reciba.
También se podría añadir la línea
\begin{Verbatim}
iptables -A OUTPUT -m state --state NEW,ESTABLISHED,RELATED -j ACCEPT
\end{Verbatim}
en el bloque \#3, para que las máquinas puedan abrir conexiones.

Como opciones avanzadas, para aceptar también tráfico DNS basta con abrir el puerto 53. La segunda propuesta es
más interesante, vamos a configurar M1 y M2 para que sólo acepten peticiones de M3. Para ello, hay que modificar
los bloques \#6 y \#7 y añadir las opciones \texttt{-s} (source) en los entrantes y \texttt{-d} (destination)
en los salientes.

\begin{Verbatim}
# 6: Permitir tráfico por el puerto 80 (HTTP) sólo de M3
iptables -A INPUT -p tcp --dport 80 -s 192.168.56.103 -j ACCEPT
iptables -A OUTPUT -p tcp --sport 80 -d 192.168.56.103 -j ACCEPT

# 7: Permitir tráfico por el puerto 443 (HTTPS) sólo de M3
iptables -A INPUT -p tcp --dport 443 -s 192.168.56.103 -j ACCEPT
iptables -A OUTPUT -p tcp --sport 443 -d 192.168.56.103 -j ACCEPT
\end{Verbatim}

Después de copiar el script a M2 y aplicar las reglas, la granja sigue funcionando sin problema. Sin embargo, no se obtiene
respuesta al conectarse desde el anfitrión.

\begin{figure}[H]
	\centering
	\includegraphics[width=140mm]{imgs/timed-out}
	\caption{Ya no podemos hacer peticiones directamente a los servidores finales, hay que usar al balanceador
		como intermediario.}
	\label{fig:timed-out}
\end{figure}

También permitiremos que nuestras máquinas puedan enviar y recibir PINGS, ahora mismo no responden (como se ve
en la Figura \ref{fig:basic}) y tampoco envían (como se ve en la siguiente figura).

\begin{figure}[H]
	\centering
	\includegraphics[width=140mm]{imgs/no-ping}
	\caption{No se pueden enviar pings.}
	\label{fig:no-ping}
\end{figure}

Las siguientes reglas habilitan los PINGS.

\begin{figure}[H]
	\centering
	\includegraphics[width=160mm]{imgs/ping}
	\caption{Ya se pueden enviar y recibir PINGS.}
	\label{fig:ping}
\end{figure}

Finalmente, mostramos como hacer que la configuración del cortafuegos se ejecute al arranque del sistema.
Hay dos formas de conseguir esto. La primera es añadir la ejecución del script con las reglas al fichero
\texttt{.bashrc}, de tal forma que esta configuración se aplica al arranque del sistema.

Otra forma de conseguir esto es la herramienta \texttt{iptables-persistent}. La instalamos con \\
\verb^sudo apt install iptables-persistent^ (no podemos acceder a internet con el cortafuegos activado).

Nos pregunta si queremos salvar la configuración actual como permanente.

\begin{figure}[H]
	\centering
	\includegraphics[width=160mm]{imgs/iptables-persistent}
	\label{fig:iptables-persistent}
\end{figure}

Podemos actualizar las reglas mediante los siguientes comandos, que salvan la
configuración actual de \emph{iptables} como permanente (tras el arranque). Hay un fichero para IPv4 y otro
para IPv6.
\begin{Verbatim}
# iptables-save > /etc/iptables/rules.v4
# ip6tables-save > /etc/iptables/rules.v6
\end{Verbatim}
(He necesitado hacerlo como root, no podía con sudo).

\end{document}
